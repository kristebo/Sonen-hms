\documentclass{artikel1}
\usepackage[utf8]{inputenc}
\usepackage[T1]{fontenc}
\usepackage[norsk]{babel}
\usepackage{textcomp}
\usepackage{multirow}
\usepackage[table,xcdraw]{xcolor}
\usepackage{amsmath}
\usepackage{float}

\title{HMS og risikoanalyse}
\begin{document}
\maketitle
\section*{inneledning}
Dette dokumentet er for deling med norske makerspaces og er produsert etter meetup på bitraf.
Sikkerhet for brukere av maskiner er svært viktig. Her er noen tips til hvordan man kan opprette rutiner og gi kunnskap om risikoene ved bruk av maskiner.


\section*{Risikoanalyse}
Risikoanalyse er en tidkrevende og viktig del av sikkerhetsarbeidet. Det er vikitg å være nøye i dette arbeidet slik at resultatene av analysen er så riktig som mulig.
Dette er vikitig siden en feil risikoanalyse kan i verste fall gjøre at man tar unødvendige/ikke tilstreklige tiltak.

Den vanligste formen for risikoanalyse er å vurdere hver risiko for dens sannsynlighet (P) og vekte konsekvensen (K). Risikoen er da $risk(x)=P(x) * K(x)$
Hvis man vekter x-verdiene fra 1-5 vi man få en risikomatrise:


\begin{table}[H]
\centering
\caption{Risikomatrise}
\label{Risikomatrise}
\begin{tabular}{cl|ccccc}
\cline{3-7}
\multicolumn{1}{l}{\cellcolor[HTML]{000000}} & \multicolumn{1}{c|}{\cellcolor[HTML]{000000}} & \multicolumn{5}{c|}{P(x)}                                                                                                                                                                  \\ \cline{3-7}
\multicolumn{1}{l}{\cellcolor[HTML]{000000}} & \cellcolor[HTML]{000000}                      & \multicolumn{1}{c|}{1}          & \multicolumn{1}{c|}{2}          & \multicolumn{1}{c|}{3}          & \multicolumn{1}{c|}{4}          & \multicolumn{1}{c|}{5}                             \\ \hline
\multicolumn{1}{|c|}{}                       & 5                                             & \cellcolor[HTML]{FFC702}moderat & \cellcolor[HTML]{F8A102}        & \cellcolor[HTML]{FD6864}        & \cellcolor[HTML]{FE0000}HØY     & \cellcolor[HTML]{CB0000}{\color[HTML]{000000} HØY} \\ \cline{2-2}
\multicolumn{1}{|c|}{}                       & 4                                             & \cellcolor[HTML]{F8FF00}moderat & \cellcolor[HTML]{FFC702}moderat & \cellcolor[HTML]{F8A102}        & \cellcolor[HTML]{FD6864}        & \cellcolor[HTML]{FE0000}HØY                        \\ \cline{2-2}
\multicolumn{1}{|c|}{}                       & 3                                             & \cellcolor[HTML]{34FF34}lav     & \cellcolor[HTML]{F8FF00}moderat & \cellcolor[HTML]{FFC702}moderat & \cellcolor[HTML]{F8A102}        & \cellcolor[HTML]{FD6864}                           \\ \cline{2-2}
\multicolumn{1}{|c|}{}                       & 2                                             & \cellcolor[HTML]{32CB00}LAV     & \cellcolor[HTML]{34FF34}lav     & \cellcolor[HTML]{F8FF00}moderat & \cellcolor[HTML]{FFC702}moderat & \cellcolor[HTML]{FFCC67}                           \\ \cline{2-2}
\multicolumn{1}{|c|}{\multirow{-5}{*}{K(x)}} & 1                                             & \cellcolor[HTML]{32CB00}LAV     & \cellcolor[HTML]{32CB00}LAV     & \cellcolor[HTML]{34FF34}lav     & \cellcolor[HTML]{F8FF00}moderat & \cellcolor[HTML]{FFC702}moderat                    \\ \cline{1-2}
\end{tabular}
\end{table}


I en risikoanalyse må man vite hva man ser etter og i hvilke perspektiv. Det er viktig å skille mellom risikoen for brukeren og maskinen. Grenseverdiene kan justeres etter behov/risiko.


\section*{Dokumentasjon}
Man kan aldri dokumentere for lite.

Her skal dokumentasjon om opplæring av bruk, avfallshåndtering, bruksanvisninger, datablad og annen informasjon være tilgjengelig. Det bør også opprettes en plan for jevnlig ettersyn av utstyr.


\section*{SOP}
Et SOP-dokument er et kortfattet og informativt oppslag som skal være i nærheten av maskinen. Denne skal inneholde informasjon om maskina, personlig verneutstyr, resultat av risikoanalysen, krav ved normal drift og krav til brukeren.



\end{document}
